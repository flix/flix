\documentclass[10pt,twocolumn]{article}
\usepackage{amsmath}
\usepackage{amssymb}
\usepackage[english]{babel}
\usepackage[utf8]{inputenc}

 
\usepackage{multicol}
\begin{document}


\section{Model-theoretic Semantics}

A Flix program $P = (C, L)$ is a set of constraints $C$ and a set of bounded lattices $L$.

A constraint is a rule $A \Leftarrow A_1, \dots, A_n$ 
where $A$ is an \emph{atom} (called the \emph{head} of the rule) 
and $A_1, \dots, A_n$ are atoms (called the \emph{body} of the rule).
A fact is a rule with an empty body.
An atom is of the form $p_\ell(e_p, t_1, \dots, t_n)$ where 
$p$ is a predicate symbol,
$\ell \in L$ is the lattice associated with $p$,
$e_p$ is a variable associated with $p$, and 
$t_1, \dots, t_n$ are terms. 
A term is either a wildcard variable, a named variable or a constant value\footnote{Note: Flix has no uninterpreted function symbols.}. 
The possible values are the unit value, the booleans (\texttt{true}, \texttt{false}),
the integers (-5, 3, 7), tagged unions of values (e.g. \texttt{Tag} $v$) and 
tuples of values (e.g. $(1, \texttt{true}. 42)$).

A bounded lattice $l \in L$ is a 6-tuple $l = (E, \bot, \top, \sqsubseteq, \sqcup, \sqcap)$ where 
$E$ is a set of elements $E \subseteq V$,
$\bot \in E$ is the least element,
$\top \in E$ is the greatest element,
$\sqsubseteq$ is the partial order on $E$,
$\sqcup$ is the least upper bound, and
$\sqcap$ is the greatest lower bound.

The \emph{Herbrand Universe} $\mathcal{U}$ of a Flix program $P$ is the set of 
all possible ground terms, i.e. the set of all values $V$.
The \emph{Herbrand Base} $\mathcal{B}$ of $P$ is the set of all possible ground atoms whose
predicate symbols occur in $P$ and where the arguments are drawn from the Herbrand Universe.

%\paragraph{Example.}
%Given the Flix program $P$:
%\begin{align*}
%&\textsl{VarPointsTo}(1, Odd()). \\
%&\textsl{VarPointsTo}(1, Even()).
%\end{align*}

An interpretation $I$ of a Flix program $P$ is a subset of the Herbrand Base $\mathcal{B}$ where 
each ground atom $p_\ell(e, v_1, \dots, v_n)$ satisfies 
$e \sqsupseteq_\ell v_1$ when $n = 1$, or 
$e \sqsupseteq_\ell [v_1 \mapsto \dots \mapsto v_n]$ when $n > 1$.

An atom $A$ is true w.r.t. an interpretation if $A \in I$. 
A conjunction of atoms $A_1, \dots, A_n$ is true w.r.t. an interpretation if each atom is true in the interpretation.
A ground rule is true if either the body conjunction is false, or the head is true.

A model $M$ of $P$ is an interpretation that makes each ground instance of a each rule in $P$ true and that satisfies the property that if $A_\ell(e_1, \dots) \in M$ and $A_\ell(e_2, \dots) \in M$ then $e_1 = e_2$ for all atoms in $M$.

A model $M_1$ is \emph{minimal} if there is no other model $M_2$ such 
that $M_2 \subset M_1$ or where $A_\ell(e_1, \dots) \in M_1$ is a ground atom in $M_1$ 
and $A_\ell(e_2, \dots) \in M_2$ is a ground atom in $M_2$ and $e_2 \sqsubset_\ell e_1$.

%\paragraph{Proof: Existence of a model?}

%\paragraph{Proof: Existence of a finite model?}

%\paragraph{Proof: Existence of a unique least model?}

\paragraph{Example.}
The Flix program $P$ with constraints:
\begin{align*}
    & A_\ell(e_A, \texttt{Even}). \\
    & A_\ell(e_A, \texttt{Odd}).
\end{align*}
and lattice $\ell = (\{\bot, \top, \texttt{Even}, \texttt{Odd}\}, \sqsubseteq, \sqcup, \sqcap)$ has
the Herbrand Universe $\mathcal{U} = \{\bot, \top, \texttt{Even}, \texttt{Odd}\}$ and 
the Herbrand Base $\mathcal{B} = $
%
\begin{align*}
    & A_\ell(\bot, \bot), && A_\ell(\bot, \texttt{Even}), \\
    & A_\ell(\bot, \texttt{Odd}), && A_\ell(\bot, \top), \\
    & A_\ell(\texttt{Even}, \bot), && A_\ell(\texttt{Even}, \texttt{Odd}), \\
    & A_\ell(\texttt{Even}, \texttt{Even}), && A_\ell(\texttt{Even}, \top), \\
    & A_\ell(\texttt{Odd}, \bot), && A_\ell(\texttt{Odd}, \texttt{Odd}), \\
    & A_\ell(\texttt{Odd}, \texttt{Even}), && A_\ell(\texttt{Odd}, \top), \\
    & A_\ell(\top, \bot), && A_\ell(\top, \texttt{Even}), \\
    & A_\ell(\top, \texttt{Odd}), && A_\ell(\top, \top)
\end{align*}
%
An interpretation is a subset of $\mathcal{B}$ such that the lattice ordering is satified. 
For example,
%
\begin{align*}
    I_1 & = \{ A_\ell(\top, \top) \} \\
    I_2 & = \{ A_\ell(\top, \top), A_\ell(\top, \texttt{Odd}) \} \\
    I_3 & = \{ A_\ell(\texttt{Even}, \texttt{Even}), A_\ell(\texttt{Odd}, \texttt{Odd}) \} \\
    I_4 & = \{ A_\ell(\top, \texttt{Even}), A_\ell(\top, \texttt{Odd}),  A_\ell(\top, \texttt{Top}) \} \\
    I_5 & = \{ A_\ell(\top, \texttt{Even}), A_\ell(\top, \texttt{Odd}) \}
\end{align*}
%
The interpretation $I_1$ is not a model of $P$ since ...

\end{document}


